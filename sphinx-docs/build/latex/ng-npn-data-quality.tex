%% Generated by Sphinx.
\def\sphinxdocclass{report}
\documentclass[letterpaper,10pt,english]{sphinxmanual}
\ifdefined\pdfpxdimen
   \let\sphinxpxdimen\pdfpxdimen\else\newdimen\sphinxpxdimen
\fi \sphinxpxdimen=.75bp\relax

\PassOptionsToPackage{warn}{textcomp}
\usepackage[utf8]{inputenc}
\ifdefined\DeclareUnicodeCharacter
% support both utf8 and utf8x syntaxes
\edef\sphinxdqmaybe{\ifdefined\DeclareUnicodeCharacterAsOptional\string"\fi}
  \DeclareUnicodeCharacter{\sphinxdqmaybe00A0}{\nobreakspace}
  \DeclareUnicodeCharacter{\sphinxdqmaybe2500}{\sphinxunichar{2500}}
  \DeclareUnicodeCharacter{\sphinxdqmaybe2502}{\sphinxunichar{2502}}
  \DeclareUnicodeCharacter{\sphinxdqmaybe2514}{\sphinxunichar{2514}}
  \DeclareUnicodeCharacter{\sphinxdqmaybe251C}{\sphinxunichar{251C}}
  \DeclareUnicodeCharacter{\sphinxdqmaybe2572}{\textbackslash}
\fi
\usepackage{cmap}
\usepackage[T1]{fontenc}
\usepackage{amsmath,amssymb,amstext}
\usepackage{babel}
\usepackage{times}
\usepackage[Bjarne]{fncychap}
\usepackage{sphinx}

\fvset{fontsize=\small}
\usepackage{geometry}

% Include hyperref last.
\usepackage{hyperref}
% Fix anchor placement for figures with captions.
\usepackage{hypcap}% it must be loaded after hyperref.
% Set up styles of URL: it should be placed after hyperref.
\urlstyle{same}
\addto\captionsenglish{\renewcommand{\contentsname}{Contents:}}

\addto\captionsenglish{\renewcommand{\figurename}{Fig.\@ }}
\makeatletter
\def\fnum@figure{\figurename\thefigure{}}
\makeatother
\addto\captionsenglish{\renewcommand{\tablename}{Table }}
\makeatletter
\def\fnum@table{\tablename\thetable{}}
\makeatother
\addto\captionsenglish{\renewcommand{\literalblockname}{Listing}}

\addto\captionsenglish{\renewcommand{\literalblockcontinuedname}{continued from previous page}}
\addto\captionsenglish{\renewcommand{\literalblockcontinuesname}{continues on next page}}
\addto\captionsenglish{\renewcommand{\sphinxnonalphabeticalgroupname}{Non-alphabetical}}
\addto\captionsenglish{\renewcommand{\sphinxsymbolsname}{Symbols}}
\addto\captionsenglish{\renewcommand{\sphinxnumbersname}{Numbers}}

\addto\extrasenglish{\def\pageautorefname{page}}

\setcounter{tocdepth}{1}



\title{ng-npn-data-quality Documentation}
\date{Apr 27, 2020}
\release{0.1}
\author{Brandon Taylor}
\newcommand{\sphinxlogo}{\vbox{}}
\renewcommand{\releasename}{Release}
\makeindex
\begin{document}

\pagestyle{empty}
\sphinxmaketitle
\pagestyle{plain}
\sphinxtableofcontents
\pagestyle{normal}
\phantomsection\label{\detokenize{index::doc}}

\phantomsection\label{\detokenize{index:module-profiler_metr}}\index{profiler\_metr (module)@\spxentry{profiler\_metr}\spxextra{module}}
Filename: profiler\_metr.py

Purpose: A collection of meteorological calculations used to 
manipulate profiler data into a form usable for analysis.

Author: Brandon Taylor

Date: 20200310

Last Modified: 20200310
\index{hypsometric() (in module profiler\_metr)@\spxentry{hypsometric()}\spxextra{in module profiler\_metr}}

\begin{fulllineitems}
\phantomsection\label{\detokenize{index:profiler_metr.hypsometric}}\pysiglinewithargsret{\sphinxcode{\sphinxupquote{profiler\_metr.}}\sphinxbfcode{\sphinxupquote{hypsometric}}}{\emph{specific\_humidities}, \emph{temperatures}, \emph{pressures}, \emph{elev}}{}
Calculates the thickness of the layer using the hypsometric equation.
Returns the resulting geometric heights in a numpy array.

\end{fulllineitems}

\index{interpolate\_uv() (in module profiler\_metr)@\spxentry{interpolate\_uv()}\spxextra{in module profiler\_metr}}

\begin{fulllineitems}
\phantomsection\label{\detokenize{index:profiler_metr.interpolate_uv}}\pysiglinewithargsret{\sphinxcode{\sphinxupquote{profiler\_metr.}}\sphinxbfcode{\sphinxupquote{interpolate\_uv}}}{\emph{interpolation\_tuple}}{}
Interpolates the two observation sets to a regular
grid.

\end{fulllineitems}

\index{pressure\_to\_height() (in module profiler\_metr)@\spxentry{pressure\_to\_height()}\spxextra{in module profiler\_metr}}

\begin{fulllineitems}
\phantomsection\label{\detokenize{index:profiler_metr.pressure_to_height}}\pysiglinewithargsret{\sphinxcode{\sphinxupquote{profiler\_metr.}}\sphinxbfcode{\sphinxupquote{pressure\_to\_height}}}{\emph{pressure}, \emph{elev}}{}
Converts pressure to height using the U.S. Standard Atmosphere,
subtracting station elevation to yield height above Mean Sea Level (MSL).

\end{fulllineitems}

\index{wind\_components() (in module profiler\_metr)@\spxentry{wind\_components()}\spxextra{in module profiler\_metr}}

\begin{fulllineitems}
\phantomsection\label{\detokenize{index:profiler_metr.wind_components}}\pysiglinewithargsret{\sphinxcode{\sphinxupquote{profiler\_metr.}}\sphinxbfcode{\sphinxupquote{wind\_components}}}{\emph{speed}, \emph{wdir\_deg}}{}
Computes the vector components of wind from speed and direction.
Wind components are return as U (east-west) and V (north-south).

\end{fulllineitems}

\index{wind\_direction() (in module profiler\_metr)@\spxentry{wind\_direction()}\spxextra{in module profiler\_metr}}

\begin{fulllineitems}
\phantomsection\label{\detokenize{index:profiler_metr.wind_direction}}\pysiglinewithargsret{\sphinxcode{\sphinxupquote{profiler\_metr.}}\sphinxbfcode{\sphinxupquote{wind\_direction}}}{\emph{u\_vec}, \emph{v\_vec}}{}
Computes the wind direction from u and v components.

\end{fulllineitems}

\index{wind\_direction\_difference() (in module profiler\_metr)@\spxentry{wind\_direction\_difference()}\spxextra{in module profiler\_metr}}

\begin{fulllineitems}
\phantomsection\label{\detokenize{index:profiler_metr.wind_direction_difference}}\pysiglinewithargsret{\sphinxcode{\sphinxupquote{profiler\_metr.}}\sphinxbfcode{\sphinxupquote{wind\_direction\_difference}}}{\emph{wdir\_hrrr}, \emph{wdir\_npn}}{}
Rotates the wind difference calculation,
so that they lie betweeen -180 and 180.

\end{fulllineitems}

\index{wind\_speed() (in module profiler\_metr)@\spxentry{wind\_speed()}\spxextra{in module profiler\_metr}}

\begin{fulllineitems}
\phantomsection\label{\detokenize{index:profiler_metr.wind_speed}}\pysiglinewithargsret{\sphinxcode{\sphinxupquote{profiler\_metr.}}\sphinxbfcode{\sphinxupquote{wind\_speed}}}{\emph{u\_vec}, \emph{v\_vec}}{}
Computes the wind speed from u and v components.

\end{fulllineitems}

\phantomsection\label{\detokenize{index:module-compare_npn_to_model}}\index{compare\_npn\_to\_model (module)@\spxentry{compare\_npn\_to\_model}\spxextra{module}}
Filename: compare\_npn\_to\_model.py

Purpose: Retrieves NPN data from either
ROCSTAR or WEATHER.GOV and compares to BUFR soundings
from either the HRRR or NAM 3-KM model.

Author: Brandon Taylor

Date: 20190411

Last Modified: 20200420
\index{available() (in module compare\_npn\_to\_model)@\spxentry{available()}\spxextra{in module compare\_npn\_to\_model}}

\begin{fulllineitems}
\phantomsection\label{\detokenize{index:compare_npn_to_model.available}}\pysiglinewithargsret{\sphinxcode{\sphinxupquote{compare\_npn\_to\_model.}}\sphinxbfcode{\sphinxupquote{available}}}{\emph{date}, \emph{icao}, \emph{hours}, \emph{overall=False}, \emph{npn\_data=False}}{}
Computes height availability

\end{fulllineitems}

\index{calc\_min\_max() (in module compare\_npn\_to\_model)@\spxentry{calc\_min\_max()}\spxextra{in module compare\_npn\_to\_model}}

\begin{fulllineitems}
\phantomsection\label{\detokenize{index:compare_npn_to_model.calc_min_max}}\pysiglinewithargsret{\sphinxcode{\sphinxupquote{compare\_npn\_to\_model.}}\sphinxbfcode{\sphinxupquote{calc\_min\_max}}}{\emph{npn\_heights}, \emph{hrrr\_heights}}{}
Calculates the height bounds across a time-series

\end{fulllineitems}

\index{compare\_profiles() (in module compare\_npn\_to\_model)@\spxentry{compare\_profiles()}\spxextra{in module compare\_npn\_to\_model}}

\begin{fulllineitems}
\phantomsection\label{\detokenize{index:compare_npn_to_model.compare_profiles}}\pysiglinewithargsret{\sphinxcode{\sphinxupquote{compare\_npn\_to\_model.}}\sphinxbfcode{\sphinxupquote{compare\_profiles}}}{}{}
endpoint for compare method

\end{fulllineitems}

\index{connect\_db() (in module compare\_npn\_to\_model)@\spxentry{connect\_db()}\spxextra{in module compare\_npn\_to\_model}}

\begin{fulllineitems}
\phantomsection\label{\detokenize{index:compare_npn_to_model.connect_db}}\pysiglinewithargsret{\sphinxcode{\sphinxupquote{compare\_npn\_to\_model.}}\sphinxbfcode{\sphinxupquote{connect\_db}}}{}{}
Create a connection to the SQLite database.
Arguments:
@return \{obj\} - sqlite3 connection object.

\end{fulllineitems}

\index{data\_availability() (in module compare\_npn\_to\_model)@\spxentry{data\_availability()}\spxextra{in module compare\_npn\_to\_model}}

\begin{fulllineitems}
\phantomsection\label{\detokenize{index:compare_npn_to_model.data_availability}}\pysiglinewithargsret{\sphinxcode{\sphinxupquote{compare\_npn\_to\_model.}}\sphinxbfcode{\sphinxupquote{data\_availability}}}{}{}
endpoint for availability method

\end{fulllineitems}

\index{data\_outages() (in module compare\_npn\_to\_model)@\spxentry{data\_outages()}\spxextra{in module compare\_npn\_to\_model}}

\begin{fulllineitems}
\phantomsection\label{\detokenize{index:compare_npn_to_model.data_outages}}\pysiglinewithargsret{\sphinxcode{\sphinxupquote{compare\_npn\_to\_model.}}\sphinxbfcode{\sphinxupquote{data\_outages}}}{}{}
endpoint for data outage tracking

\end{fulllineitems}

\index{data\_outages\_metadata() (in module compare\_npn\_to\_model)@\spxentry{data\_outages\_metadata()}\spxextra{in module compare\_npn\_to\_model}}

\begin{fulllineitems}
\phantomsection\label{\detokenize{index:compare_npn_to_model.data_outages_metadata}}\pysiglinewithargsret{\sphinxcode{\sphinxupquote{compare\_npn\_to\_model.}}\sphinxbfcode{\sphinxupquote{data\_outages\_metadata}}}{}{}
endpoint for data outage tracking metadata including icao and dates

\end{fulllineitems}

\index{difference() (in module compare\_npn\_to\_model)@\spxentry{difference()}\spxextra{in module compare\_npn\_to\_model}}

\begin{fulllineitems}
\phantomsection\label{\detokenize{index:compare_npn_to_model.difference}}\pysiglinewithargsret{\sphinxcode{\sphinxupquote{compare\_npn\_to\_model.}}\sphinxbfcode{\sphinxupquote{difference}}}{\emph{user\_params}, \emph{hourly='t'}, \emph{overall=False}, \emph{npn\_data=False}, \emph{raob=False}, \emph{qc=False}}{}
Computes difference between NPN data and HRRR data by interpolating to 
regular height levels, starting at 100 meters, going to 10 km, at 100 meter intervals.

\end{fulllineitems}

\index{difference\_profiles() (in module compare\_npn\_to\_model)@\spxentry{difference\_profiles()}\spxextra{in module compare\_npn\_to\_model}}

\begin{fulllineitems}
\phantomsection\label{\detokenize{index:compare_npn_to_model.difference_profiles}}\pysiglinewithargsret{\sphinxcode{\sphinxupquote{compare\_npn\_to\_model.}}\sphinxbfcode{\sphinxupquote{difference\_profiles}}}{}{}
endpoint for difference method

\end{fulllineitems}

\index{extra\_B3() (in module compare\_npn\_to\_model)@\spxentry{extra\_B3()}\spxextra{in module compare\_npn\_to\_model}}

\begin{fulllineitems}
\phantomsection\label{\detokenize{index:compare_npn_to_model.extra_B3}}\pysiglinewithargsret{\sphinxcode{\sphinxupquote{compare\_npn\_to\_model.}}\sphinxbfcode{\sphinxupquote{extra\_B3}}}{\emph{date}, \emph{icao}, \emph{hours}}{}
reads Build 3 data from 2017

\end{fulllineitems}

\index{extra\_heights() (in module compare\_npn\_to\_model)@\spxentry{extra\_heights()}\spxextra{in module compare\_npn\_to\_model}}

\begin{fulllineitems}
\phantomsection\label{\detokenize{index:compare_npn_to_model.extra_heights}}\pysiglinewithargsret{\sphinxcode{\sphinxupquote{compare\_npn\_to\_model.}}\sphinxbfcode{\sphinxupquote{extra\_heights}}}{\emph{date}, \emph{icao}, \emph{hours}, \emph{hourly}}{}
Tests the extra heights algorithmn

\end{fulllineitems}

\index{generate\_expected\_dates() (in module compare\_npn\_to\_model)@\spxentry{generate\_expected\_dates()}\spxextra{in module compare\_npn\_to\_model}}

\begin{fulllineitems}
\phantomsection\label{\detokenize{index:compare_npn_to_model.generate_expected_dates}}\pysiglinewithargsret{\sphinxcode{\sphinxupquote{compare\_npn\_to\_model.}}\sphinxbfcode{\sphinxupquote{generate\_expected\_dates}}}{\emph{start\_date\_str}, \emph{end\_date\_str}, \emph{hourly}}{}
Generates expected dates for data outage tracking purposes.

\end{fulllineitems}

\index{hourly() (in module compare\_npn\_to\_model)@\spxentry{hourly()}\spxextra{in module compare\_npn\_to\_model}}

\begin{fulllineitems}
\phantomsection\label{\detokenize{index:compare_npn_to_model.hourly}}\pysiglinewithargsret{\sphinxcode{\sphinxupquote{compare\_npn\_to\_model.}}\sphinxbfcode{\sphinxupquote{hourly}}}{\emph{npn\_data}, \emph{hours}}{}
Returns the percentage availability of hourly data encountered,
from given expected number of hours.

\end{fulllineitems}

\index{index\_html() (in module compare\_npn\_to\_model)@\spxentry{index\_html()}\spxextra{in module compare\_npn\_to\_model}}

\begin{fulllineitems}
\phantomsection\label{\detokenize{index:compare_npn_to_model.index_html}}\pysiglinewithargsret{\sphinxcode{\sphinxupquote{compare\_npn\_to\_model.}}\sphinxbfcode{\sphinxupquote{index\_html}}}{}{}
Sends main page static HTML

\end{fulllineitems}

\index{model() (in module compare\_npn\_to\_model)@\spxentry{model()}\spxextra{in module compare\_npn\_to\_model}}

\begin{fulllineitems}
\phantomsection\label{\detokenize{index:compare_npn_to_model.model}}\pysiglinewithargsret{\sphinxcode{\sphinxupquote{compare\_npn\_to\_model.}}\sphinxbfcode{\sphinxupquote{model}}}{}{}
endpoint for model check method

\end{fulllineitems}

\index{model\_check() (in module compare\_npn\_to\_model)@\spxentry{model\_check()}\spxextra{in module compare\_npn\_to\_model}}

\begin{fulllineitems}
\phantomsection\label{\detokenize{index:compare_npn_to_model.model_check}}\pysiglinewithargsret{\sphinxcode{\sphinxupquote{compare\_npn\_to\_model.}}\sphinxbfcode{\sphinxupquote{model\_check}}}{\emph{date}, \emph{icao}, \emph{hours}, \emph{variable}}{}
Computes difference between NPN data and HRRR data by interpolating to 
regular height levels, starting at 100 meters, going to 10 km, at 100 meter intervals.

\end{fulllineitems}

\index{overview() (in module compare\_npn\_to\_model)@\spxentry{overview()}\spxextra{in module compare\_npn\_to\_model}}

\begin{fulllineitems}
\phantomsection\label{\detokenize{index:compare_npn_to_model.overview}}\pysiglinewithargsret{\sphinxcode{\sphinxupquote{compare\_npn\_to\_model.}}\sphinxbfcode{\sphinxupquote{overview}}}{}{}
endpoint for dashboard overview

\end{fulllineitems}

\index{profile\_html() (in module compare\_npn\_to\_model)@\spxentry{profile\_html()}\spxextra{in module compare\_npn\_to\_model}}

\begin{fulllineitems}
\phantomsection\label{\detokenize{index:compare_npn_to_model.profile_html}}\pysiglinewithargsret{\sphinxcode{\sphinxupquote{compare\_npn\_to\_model.}}\sphinxbfcode{\sphinxupquote{profile\_html}}}{}{}
Sends profile comparison static HTML

\end{fulllineitems}

\index{read\_b3\_bufr() (in module compare\_npn\_to\_model)@\spxentry{read\_b3\_bufr()}\spxextra{in module compare\_npn\_to\_model}}

\begin{fulllineitems}
\phantomsection\label{\detokenize{index:compare_npn_to_model.read_b3_bufr}}\pysiglinewithargsret{\sphinxcode{\sphinxupquote{compare\_npn\_to\_model.}}\sphinxbfcode{\sphinxupquote{read\_b3\_bufr}}}{\emph{fname}}{}
Reads NPN Build 3 style BUFR files.
Extracts and converts height and wind speed/direction.

\end{fulllineitems}

\index{read\_ncep\_bufr() (in module compare\_npn\_to\_model)@\spxentry{read\_ncep\_bufr()}\spxextra{in module compare\_npn\_to\_model}}

\begin{fulllineitems}
\phantomsection\label{\detokenize{index:compare_npn_to_model.read_ncep_bufr}}\pysiglinewithargsret{\sphinxcode{\sphinxupquote{compare\_npn\_to\_model.}}\sphinxbfcode{\sphinxupquote{read\_ncep\_bufr}}}{\emph{fname}, \emph{convert\_uv}, \emph{offset}}{}
Reads NCEP BUFR type files,
which include the local table as the first message.
Extracts and converts height and wind speed/direction.

\end{fulllineitems}

\index{read\_npn\_csv() (in module compare\_npn\_to\_model)@\spxentry{read\_npn\_csv()}\spxextra{in module compare\_npn\_to\_model}}

\begin{fulllineitems}
\phantomsection\label{\detokenize{index:compare_npn_to_model.read_npn_csv}}\pysiglinewithargsret{\sphinxcode{\sphinxupquote{compare\_npn\_to\_model.}}\sphinxbfcode{\sphinxupquote{read\_npn\_csv}}}{\emph{fname}}{}
Reads NGNPN CSV files.
Extracts and converts height and wind speed/direction.

\end{fulllineitems}

\index{retrieve\_hrrr\_winds() (in module compare\_npn\_to\_model)@\spxentry{retrieve\_hrrr\_winds()}\spxextra{in module compare\_npn\_to\_model}}

\begin{fulllineitems}
\phantomsection\label{\detokenize{index:compare_npn_to_model.retrieve_hrrr_winds}}\pysiglinewithargsret{\sphinxcode{\sphinxupquote{compare\_npn\_to\_model.}}\sphinxbfcode{\sphinxupquote{retrieve\_hrrr\_winds}}}{\emph{date}, \emph{hour}, \emph{user\_params\_dict}, \emph{offset=0}, \emph{convert\_uv=False}}{}
Reads in hrrr bufr sounding from local archive

\end{fulllineitems}

\index{retrieve\_npn\_winds() (in module compare\_npn\_to\_model)@\spxentry{retrieve\_npn\_winds()}\spxextra{in module compare\_npn\_to\_model}}

\begin{fulllineitems}
\phantomsection\label{\detokenize{index:compare_npn_to_model.retrieve_npn_winds}}\pysiglinewithargsret{\sphinxcode{\sphinxupquote{compare\_npn\_to\_model.}}\sphinxbfcode{\sphinxupquote{retrieve\_npn\_winds}}}{\emph{user\_params}, \emph{hourly='t'}, \emph{hours=24}}{}
Reads in npn data from ROCSTAR.
Tries the primary ROCSTAR server first with timeout,
then tries the backup ROCSTAR server.

\end{fulllineitems}

\index{retrieve\_raob() (in module compare\_npn\_to\_model)@\spxentry{retrieve\_raob()}\spxextra{in module compare\_npn\_to\_model}}

\begin{fulllineitems}
\phantomsection\label{\detokenize{index:compare_npn_to_model.retrieve_raob}}\pysiglinewithargsret{\sphinxcode{\sphinxupquote{compare\_npn\_to\_model.}}\sphinxbfcode{\sphinxupquote{retrieve\_raob}}}{\emph{date}, \emph{hour\_str}, \emph{icao}}{}
Reads in npn data from ROCSTAR.
Tries the primary ROCSTAR server first with timeout,
then tries the backup ROCSTAR.

\end{fulllineitems}

\index{sqlite\_date\_parse() (in module compare\_npn\_to\_model)@\spxentry{sqlite\_date\_parse()}\spxextra{in module compare\_npn\_to\_model}}

\begin{fulllineitems}
\phantomsection\label{\detokenize{index:compare_npn_to_model.sqlite_date_parse}}\pysiglinewithargsret{\sphinxcode{\sphinxupquote{compare\_npn\_to\_model.}}\sphinxbfcode{\sphinxupquote{sqlite\_date\_parse}}}{\emph{date}}{}
Returns a data in hyphenated format for SQLite purposes.
Example: input 20200101, output 2020-01-01

\end{fulllineitems}

\index{track\_html() (in module compare\_npn\_to\_model)@\spxentry{track\_html()}\spxextra{in module compare\_npn\_to\_model}}

\begin{fulllineitems}
\phantomsection\label{\detokenize{index:compare_npn_to_model.track_html}}\pysiglinewithargsret{\sphinxcode{\sphinxupquote{compare\_npn\_to\_model.}}\sphinxbfcode{\sphinxupquote{track\_html}}}{}{}
Sends data outages static HTML

\end{fulllineitems}



\chapter{Indices and tables}
\label{\detokenize{index:indices-and-tables}}\begin{itemize}
\item {} 
\DUrole{xref,std,std-ref}{genindex}

\item {} 
\DUrole{xref,std,std-ref}{modindex}

\item {} 
\DUrole{xref,std,std-ref}{search}

\end{itemize}


\renewcommand{\indexname}{Python Module Index}
\begin{sphinxtheindex}
\let\bigletter\sphinxstyleindexlettergroup
\bigletter{c}
\item\relax\sphinxstyleindexentry{compare\_npn\_to\_model}\sphinxstyleindexpageref{index:\detokenize{module-compare_npn_to_model}}
\indexspace
\bigletter{p}
\item\relax\sphinxstyleindexentry{profiler\_metr}\sphinxstyleindexpageref{index:\detokenize{module-profiler_metr}}
\end{sphinxtheindex}

\renewcommand{\indexname}{Index}
\printindex
\end{document}